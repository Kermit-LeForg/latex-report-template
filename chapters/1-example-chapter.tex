\chapter{Chapter 1 name...}


\section{Code Blocks}

\subsection{C-Sharp Code Block}

\begin{figure}[ht]
    %\centering
    \begin{minted}[xleftmargin=30pt,linenos]{csharp}
        public class Program
        {
            public static void Main(string[] args)
            {
                Console.WriteLine("Hello World!");
            }
        }
    \end{minted}
    \caption{C-Sharp Code Block}
    \label{fig:CSharpCodeBlock}
\end{figure}

\subsection{Python Code Block}

You can print to the python standard output using the \mintinline{python}|print| keyword. 

\begin{figure}[ht]
    \centering
    \begin{lstlisting}[language=python]
        print("Hello World!, this is listings instead of minted")
    \end{lstlisting}
    \caption{Python Code Block}
    \label{fig:PythonCodeBlock}
\end{figure}


\begin{figure}[ht]
    \centering
    \begin{Verbatim}
        print("this is as-is")
    \end{Verbatim}
    \caption{Mother Mother}
    \label{fig:berbatim}
\end{figure}

\newpage


\section{Maths}

\begin{figure}[ht]
    \centering
    \begin{math}
        E=mc^2 
    \end{math}
    \caption{Einstein Formula}
    \label{fig:Emc2}

\end{figure}

\begin{figure}[ht]
    \centering
    \begin{math}
        \frac{1}{2} 
    \end{math}
    \caption{Fraction}
    \label{fig:Fraction 1}

\end{figure}

\subsection{this is a subsection}

\lipsum[7-8]

\subsection{another subection}

\lipsum[4-6]
