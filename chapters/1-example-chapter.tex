% !TeX root = ..\report.tex
\chapter{Chapter 1 Name}

\section{Code}

\subsection{Code Block}

\begin{figure}[H]
    \begin{minted}[breaklines, linenos]{csharp}
namespace Project;

public class Program
{
    public static void Main(string[] args)
    {
        Console.WriteLine("Hello World!");
    }
}
    \end{minted}
    \macFC{C\# Code Block}{code:code_block}
\end{figure}

All labels used for code blocks should begin with `code:'

\subsection{In-Line Code}

You can print to the python standard output using the \mintinline{python}|print| keyword. 

\newpage
\section{Figures}

Using \mintinline{tex}|\macLFC| will allow you to have a different caption below the image than is used in the list of figures.
This can be used when the caption requires a reference but you don't want the reference in the list of figures.

\begin{figure}[H]
    \centering
    \includegraphics[width=\textwidth]{assets/example-asset.jpg}
    \macLFC{Caption for ToC}{Caption Below}{fig:figure1}
\end{figure}

\section{Tables}
    
\begin{table}[H]
    \centering
    \begin{tabular}{||c|c|c|c||} 
     \hline
     Col1 & Col2 & Col2 & Col3 \\ [0.5ex] 
     \hline\hline
     1 & 6 & 87837 & 787 \\ 
     2 & 7 & 78 & 5415 \\
     3 & 545 & 778 & 7507 \\
     4 & 545 & 18744 & 7560 \\
     5 & 88 & 788 & 6344 \\ [1ex] 
     \hline
    \end{tabular}
    \macFC{Table to test captions and labels.}{table:example}
\end{table}

\newpage
\section{Maths}

Using the macro \mintinline{tex}|\macMath| you can write a maths block and include a caption and label, prefixed with `math:'

\macMath{
    E = mc^2
}
{Einstein Formula}
{math:emc2}

\section{Acronyms}

When first using acronyms, it will display the full phrase and the acronym in brackets.
After that, it will only display the acronym.
\par 


\begin{figure}[H]
    \begin{minted}[xleftmargin=30pt,linenos]{tex}
\ac{ex} \\ 
\ac{ex}
    \end{minted}
    \macFC{TeX Code Block}{tex}
\end{figure}

\noindent
\ac{ex} \\
\ac{ex}

\section{Referencing}

To reference use the macro \mintinline{tex}|\citep{}|.
Insert the unique keyword given in the \emph{ref.bib} file.
When referencing a company or group, encase the author value in two sets of curly braces.
Finally, ensure the formatting is correct using \mintinline{tex}|\url| and \mintinline{tex}|\emph{}|