\chapter{Chapter 1 name...}


\section{Code Blocks}

\subsection{C-Sharp Code Block}

\begin{figure}[ht]
    %\centering
    \begin{minted}[xleftmargin=30pt,linenos]{csharp}
        public class Program
        {
            public static void Main(string[] args)
            {
                Console.WriteLine("Hello World!");
            }
        }
    \end{minted}
    \macFC{CSharp Code Block}{cscb}
\end{figure}


\subsection{Python Code Block}

You can print to the python standard output using the \mintinline{python}|print| keyword. 

\newpage


\section{Maths}

\macMath{
    E = mc^2
} 
{Einstein Formula}
{emc2}

\subsection{this is a subsection}

\lipsum[7-8]

\subsection{another subection with a table in it}


\begin{table}[h!]
\centering
\begin{tabular}{||c|c|c|c||} 
 \hline
 Col1 & Col2 & Col2 & Col3 \\ [0.5ex] 
 \hline\hline
 1 & 6 & 87837 & 787 \\ 
 2 & 7 & 78 & 5415 \\
 3 & 545 & 778 & 7507 \\
 4 & 545 & 18744 & 7560 \\
 5 & 88 & 788 & 6344 \\ [1ex] 
 \hline
\end{tabular}
\macFC{Table to test captions and labels.}{table:example}
\end{table}

\section{Acronyms}

When first using acronyms, it will display the full phrase and the acronym in brackets. After that, it will only display the acronym.\\

\begin{figure}[ht]
    \begin{minted}[xleftmargin=30pt,linenos]{tex}
        \ac{ex} \\ 
        \ac{ex}
    \end{minted}
    \macFC{TeX Code Block}{tex}
\end{figure}

\noindent
\ac{ex} \\
\ac{ex}